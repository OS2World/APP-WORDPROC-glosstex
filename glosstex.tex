% -*- latex -*-
% $Id: glosstex.tex,v 1.22 1997/05/04 12:19:52 yavuzv Exp $

\def\fileversion{0.2}
\def\filedate{1997/05/04}
\documentclass[draft,a4paper]{ltxdoc}

%\usepackage[latin1]{inputenc}
%\usepackage[T1]{fontenc}
%\usepackage[german]{babel}
\usepackage{shortvrb}\MakeShortVerb{\|}
\usepackage{glosstex}
\makeglossary

\setlength{\arrayrulewidth}{0.4pt}
\setlength{\doublerulesep}{0.3pt}

\newcommand{\unix}{{\scshape Unix}}
\newcommand{\MakeIndex}{{MakeIndex}}
\newcommand{\GloTeX}{Glo\TeX}
\newcommand{\darg}[1]{{\itshape #1\/}}
\newcommand{\dopt}[1]{$[$#1$]$}

\begin{document}
\title{\GlossTeX{} \fileversion}
\author{\scshape Volkan Yavuz\footnote{e-mail: yavuzv@rumms.uni-mannheim.de}}
\date{\filedate}
\maketitle

%%
%% Abstract

\begin{abstract}
  \GlossTeX{} is a tool for the preparation of glossaries. It greatly
  simplifies this task. One or more glossary-definition files serve as
  databases which contain the descriptions of terms. These terms are
  identified through labels. Based upon these labels set into the
  \TeX{}-source, \GlossTeX{} determines which entries have to appear in
  the glossary. \GlossTeX{} uses \MakeIndex{} for the sorting of the
  glossary. References to the place where a term appears in the text
  can be set in the glossary.
\end{abstract}

\tableofcontents
%\listoffigures
\listoftables

%%
%% Introduction

\section{Introduction}
\subsection{Purpose}
\GlossTeX{} is a tool for the preparation of glossaries. Based upon the
labels set into the \TeX{}-source, \GlossTeX{} determines which entries
from a glossary-definition file have to be processed. \GlossTeX{} then
creates an intermediate file that has to be processed by \MakeIndex{} for
sorting. The output of \MakeIndex{}  is then included into the
\TeX{}-source for typesetting. This whole process can be automated by
using a makefile.

\subsection{History}
I created \GlossTeX{} because there were no tools for the preparation
of glossaries that fit my needs. \GlossTeX{} is mainly a combination of
the features of {\sffamily nomencl} and \GloTeX{}. The way \GlossTeX{} 
handles page references is almost identical to the way {\sffamily
  nomencl} does it. The use of glossary databases is inspired by
\GloTeX{}. In fact, \GlossTeX{} should be able to read any glossary
database that \GloTeX\ is able to read, and vice versa. Some code to
parse the database file is taken directly from the \GloTeX\ sources.

\subsection{Legalese}
\GlossTeX{} is provided ``as is'' and comes with absolutely no
warranty. It is covered by the GNU General Public License (see the
file |COPYING| that comes with this package).

\begin{center}
  \noindent\copyright\ Volkan Yavuz, 1997
\end{center}


%%
%% Installation

\section{Installation}
You need \LaTeX2e{}, \MakeIndex{} and a C compiler to build and install
\GlossTeX{}. Just type |make| to compile \GlossTeX{}. (If you don't
have ``make'', just compile each |.c|-file and link all the resulting
files together.) So, these commands should be sufficient to build
\GlossTeX{}
%
\begin{quote}
\begin{verbatim}
cc -c database.c -o database.o
cc -c error.c -o error.o
cc -c labels.c -o labels.o
cc -c list.c -o list.o
cc -c main.c -o main.o
cc -c version.c -o version.o
cc database.o error.o labels.o list.o main.o version.o -o glosstex
\end{verbatim}
\end{quote}
%

Put the final binary (|glosstex| or |glosstex.exe|) somewhere in your
path and copy |glosstex.sty| to a place \TeX\ looks for \LaTeX{}-files
(e.\,g.~|texmf/|\-|tex/|\-|latex/|\-|misc|). You also need a
|.ist|-file\glosstex[a]{ist-file} for \MakeIndex{} (see |glosstex.ist|)
and a glossary-definition database for \GlossTeX{} (see
|glosstex.gdf|). Unfortunately, \GlossTeX{} doesn't have any
path-searching capabilities (yet), so you have to specify the whole
path for any |.glo| and |.gdf|-files\glosstex[a]{gdf-file}.

%%
%% Usage

\section{Usage}
To tell \LaTeX\ that we want to generate a glossary, the command
%
\begin{quote}
  |\makeglossary|
\end{quote}
%
is issued in the preamble of the document. The \LaTeX{}-macros needed
by \GlossTeX{} have to be included into the source using
%
\begin{quote}
|\usepackage|\dopt{|[|\darg{options}|]|}|{glosstex}|
\end{quote}
%
where \darg{options} may be one of |refpage| or |norefpage|. The
latter one is the default. These options are described later in
section~\ref{sec:page_references}.

Whenever you want a term to appear in the glossary, you insert
%
\begin{quote}
  |\glosstex|\dopt{|[|\darg{mode}|]|}|{|\darg{label}|}|
\end{quote}
%
into the text (preferably directly after the term to make sure that
page references are right). \darg{label} references the entry and the
optional argument \darg{mode} determines the mode for page references.
It works in conjunction with the optional arguments |refpage| and
|norefpage| to the package. \darg{mode} defaults to |p|.

This writes a line of the form
%
\begin{quote}
|\glosstexentry{|\darg{label}|}{|\darg{mode}|}{|\darg{page}|}|  
\end{quote}
%
into the |.glo|-file. \darg{label} and \darg{mode} are the
values supplied to the |\glosstex| command in the source. This file
has then to be processed by \GlossTeX{} and \MakeIndex{} subsequently
to produce a complete glossary-file (suffix |.gls|) which is finally
included into the document. This may be done using the commands
%
\begin{quote}
\begin{verbatim}
\printglosstex
\end{verbatim}
\end{quote}

\subsection{Glossary-Definition File}
A glossary-definition file (suffix |.gdf|) is needed which serves
as a database for \GlossTeX{}, holding the actual descriptions of all
entries. This file contains entries of the form
%
\begin{quote}
|@entry{|\darg{label }\dopt{|,|\darg{item}}|}| \darg{text}
\end{quote}
%
where \darg{label} and \darg{text} are both mandatory elements. \darg{label}
is used to identify the entry and \darg{text} may contain any amount
of \TeX{}-source, being the actual definition of the item. The
optional argument \darg{item} describes the appearance of the item in
the glossary. If omitted, it defaults to \darg{label}. It can be used
when some special form of typesetting is wanted.

This is an example for a valid |.gdf|-file. Note that all lines
until the first line starting with |@entry{| are ignored. Thus they
can serve a comments.
%
\begin{quote}    
\begin{verbatim}
# Heading lines not starting with '@entry{' are ignored

@entry{gdf-file, \texttt{gdf}-file} Database file
containing glossary-definitions for Gloss\TeX{}.
\end{verbatim}
\end{quote}
%
\darg{item} may contain any number of ``|{|''; as long as they are
balanced, \GlossTeX{} will be able to correctly parse the line. 

You may have noted that the format of the |.gdf|-file is identical to
the format \GloTeX\ is using. \GlossTeX{} should be able to read any
files that \GloTeX\ reads, and vice versa. In fact, the algorithms
that parse the various input files are very similar and the one that
parses \darg{item} is taken directly from the sources to \GloTeX{}.

\subsection{Invocation}
After the first run of \LaTeX{}, the |.glo|-file contains all
necessary information for the preparation of the glossary. \GlossTeX\
is then invoked to read one or more |.gdf|-files and outputs all
definitions that are referenced in the |.glo|-file. The output of
\GlossTeX{} is then processed by \MakeIndex{}  for sorting.

\GlossTeX{} is invoked in a \unix{}-like environment using the
following command
%
\begin{quote}
  |glosstex |\darg{glo-file}| |\darg{gdf-file}| |\dopt{\darg{gdf-file}\dopt{...}} \dopt{|-o |\darg{gxs-file}} \dopt{|-t |\darg{gxg-file}}
\end{quote}

Any number of |.gdf|-files can be specified. If the output-file
(|.gxs|\glosstex[a]{gxs-file}) is omitted, it's name is derived from
the |.glo|-file. The same applies for the log-file (|.gxg|). If the
specified files contain no extension (i.\,e.~they contain no ``|.|'')
the standard extensions are added. Thus the command
%
\begin{quote}
  |glosstex thesis thesis.foo master -t glosstex.log|
\end{quote}
%
would open the file |thesis.glo| for input and use the files
|master.gdf| and |thesis.foo| as glossary-databases. The output would
be written into |thesis.gxs| and the log-file would be named
|glosstex.log|.

The |.gxs|-file then looks like
%
\begin{quote}
\begin{verbatim}
\glosstexentry{gdf-file@[\texttt{gdf}-file] Database file
containing glossary-definitions for Gloss\TeX{}.
\GlossTeXMode{p}|GlossTeXPage}{3}
\end{verbatim}
\end{quote}

This file is then processed my \MakeIndex{}  yielding the final
output. \MakeIndex{}  has to be invoked in this way
%
\begin{quote}
  |makeindex | \darg{gxs-file} |-o | \darg{glx-file} |-s | \darg{ist-file} \dopt{|-t |\darg{glg-file}}
\end{quote}

It is very important to use an appropriate |.ist|-file for \MakeIndex\
to be able to read and write files in the correct format. \GlossTeX\
comes with |glosstex.ist| which can be used as style-file for \MakeIndex{}.

The command
%
\begin{quote}
|makeindex thesis.gxs -o thesis.glx -s glosstex.ist|
\end{quote}
%
produces the final |.gls|-file which is then include by
|\printglosstex| during the next \LaTeX{}-run.

\subsection{Page References}
\label{sec:page_references}
You may want a reference in the glossary to the place where the term
first appears in the text. This can be done using the optional
argument to |\glosstex|. That argument in combination with the option
to |\usepackage| controls these references. Table~\ref{tab:references} gives
an overview of all possible combinations of these two arguments. A
``$\times$'' indicates that a reference is produced.

\begin{table}[htbp]
  \begin{center}
    \leavevmode
    \begin{tabular}{rcc}
      \hline\hline
      & |refpage| & |norefpage|\\
      \hline
      never |[n]| &$-$ & $-$ \\
      package |[p]| & $\times$ & $-$\\
      always |[a]| & $\times$ & $\times$\\
      \hline\hline
    \end{tabular}
    \caption{Options controlling the appearance of page references.}
    \label{tab:references}
  \end{center}
\end{table}

One possible usage of this feature: while debugging a document, turn on
page references by using the option |refpage| to the page. Every
glossary entry included with the modes |[a]| (always) or the default |[p]|
(package) will contain a reference. After debugging, remove the option
|refpage| and only those entries that were included with mode |[a]| will
still have a reference.

\subsection{Cross-References}
It may be useful to use cross-references in glossary entries. Assume
you have referenced |\glosstex{ascii}| which describes the term
ASCII\glosstex{ascii}. You may also want to include EBCDIC as an
example for another character encoding. To achieve this, write this
into the definition of ASCII
%
\begin{verbatim}
See also EBCDIC\glosstexxref{ebcdic}. 
\end{verbatim}
%
and \GlossTeX{} then produces ``See also EBCDIC'' and also includes
the definition for EBCDIC into the glossary.

\subsection{\GlossTeX{} and \textsf{nomencl}}
It is possible to use {\sffamily nomencl} and \GlossTeX{} in one
document without problems. Although both use the |.glo|-file,
\GlossTeX{} will ignore all lines from |.glo| not starting with
``|\glosstexentry{|''. Using \MakeIndex{} with the proper |nomencl.ist|
file in turn ignores all |\glosstexentr|ie|{|s. 

The following commands show how to deal with documents using both
\GlossTeX{} and \textsf{nomencl}.
\begin{verbatim}
latex thesis
glosstex thesis.glo thesis.gdf -o thesis.gxs -t thesis.gxg
makeindex thesis.gxs -o thesis.glx -t thesis.glg -s glosstex.ist
makeindex thesis.glo -o thesis.gls -t thesis.nlg -s nomencl.ist
latex thesis
\end{verbatim}

\LaTeX\ produces the |.glo|-file which serves as input to both
\GlossTeX{} and \textsf{nomencl}. \GlossTeX{} produces the |.gxs|-file
which is processed by \MakeIndex{} using |glosstex.ist|, yielding the
|.glx|-file. This file is included with |\printglosstex|. To produce
the nomenclature, the same |.glo|-file is processed by \MakeIndex{},
this time using |nomencl.ist|. This yields the |.glo|-file which is
included using |\printglossary|. Confused? Read it again. Easy, isn't
it?
    
%%
%% Customizing

\section{Customising}
\GlossTeX{} can be customised using a file named |glosstex.cfg|

If you're writing a german document, you may want to change the text
that \GlossTeX{} is using for page reference. Instead of ``(See page
1.)'' you want ``(Siehe Seite 1.)''. Additionally, you want to use
"`Glossar"' instead of "`Glossary"'. Put the following lines into the
file |glosstex.cfg|
%
\begin{verbatim}
  \renewcommand{\glosstexpage}[1]{(Siehe Seite #1.)}
  \renewcommand{\glosstexname}{\section*{Glossar}}
\end{verbatim}

%%
%% Some Details

\section{Some Details}
While reading the |.glo|-file, \GlossTeX{} only considers the first
appearance of one label. All subsequent entries are silently ignored.
(Almost silently, because the |.gxg|-file will contain detailed
information about this, and more.)

The same applies while reading one or more |.gdf|-files. The first
entry wins, all other entries are ignored. This fact can be utilised
in some way. Assume you have a |master.gdf| which contains general
terms and a file |thesis.gdf| which only contains terms that are
intended for use in your thesis. Whenever an entry is present in both
|.gdf|-files, the one from |thesis.gdf| should be taken. To achieve
this, specify |thesis.gdf| \emph{before} |master.gdf|.

The |.gxg|-file contains additional information, e.\,g.~when no entry
was found for a label.

After \GlossTeX{} is finished, it will print some statistics about read
labels, unresolved entries and the like. More detailed information can
then be found in the |.gxg|-file.

%%
%% Bugs and Features

\section{Bugs, Features and Things To-Do}
I chose to use |\glosstex| instead of |\glossary| to avoid conflicts
with other packages that may use a glossary-file. The document classes
``ltxdoc'' and ``doc'' use the glossary for typesetting the change
history. 

\GlossTeX{} should be able to use an environment string which specifies
the directory to search for |.gdf|-files. This will probably be added
sometime, maybe even using kpathsea.

\GlossTeX{} should compile in any ANSI C environment. I have built
\GlossTeX{} on Linux 2.0.0 with gcc 2.7.2, make 3.7.4 and libc
5.2.17. It also compiled out-of-the box on DOS using DJGPP
2.6.3. There is a OS/2 port done by Stefan A. Deutscher
(\texttt{sad@utk.edu}) on CTAN in |support/|\-|os2/|\-|binaries|. If
you have successfully built \GlossTeX{} on some other platform (like VMS
or Macintosh) please contact me, so I can enhance \GlossTeX{}.

\printglosstex

\end{document}

%%% Local Variables: 
%%% mode: latex
%%% mode: outline-minor
%%% TeX-master: t
%%% End: 
